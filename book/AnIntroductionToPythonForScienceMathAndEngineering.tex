%begin prelude
\documentclass{report}
\usepackage{graphicx} %for importing pictures
\usepackage{float}
\usepackage{alltt} %extended verbatim mode
\usepackage{fullpage}
\renewcommand{\ttdefault}{txtt} %bold in alltt
\usepackage{color}
\usepackage{parskip} %no idents on paragraphs

  \usepackage[T1]{fontenc} %to make < and > brackets work in text mode
  \usepackage{lmodern}  

\usepackage{hyperref}
\usepackage{glossaries}

%glossary
\newglossaryentry{class}
{
    name=class,
    description={In object-oriented programming, a class is a construct that is used as a blueprint to create instances of the class (class instances, class objects, instance objects or just objects). \cite{website:wikipedia} }
}

\newglossaryentry{package manager}
{
    name=package manager,
    description={A collection of software tools to automate the process of installing, upgrading, configuring, and removing software packages for a computer's operating system in a consistent manner. \cite{website:wikipedia} }
}

\newglossaryentry{ide}
{
    name=IDE,
    description={Integrated Development Environment. A software application that provides comprehensive facilities to computer programmers for software development.\cite{website:wikipedia} },
    first=integrated development environment or IDE
}

\newglossaryentry{repl}
{
    name=REPL,
    description={Read Evaluate Print Loop. A simple, interactive computer programming environment. The term is most usually used to refer to a Lisp interactive environment, but can be applied to command line shells and similar environments for Smalltalk, Standard ML, Perl, Prolog, Scala, Python, Ruby, Haskell, APL, BASIC, J, Tcl, and other languages as well. \cite{website:wikipedia} },
    first=read-evaluate-print loop or REPL
}

%end glossary

\makeglossaries

%end prelude

\begin{document}
\title{An Introduction to Python For Science, Math and Engineering}
\author{Nathan Alday}
\date{\today}
\maketitle

\chapter{Introduction}
Python is a very flexible, powerful and easy to learn and use programming language that is widely applicable to both numerical analysis and general programming tasks. It's also Free -- both costing nothing and free to use, modify and copy with no licensing restrictions to speak of. It can replace Matlab, C, C++ and most other languages for a wide variety of tasks, both at the scripting level and the large project level,  while lessening development time and debugging and often offering comparable or greater performance. It's also very good at interfacing with other languages, particularly C and C++. 

The language is cross platform (it even runs on Android), has an enormous and well supported set of standard libraries as well a an active open source community that has developed a large ecosystem of Python packages that we can leverage and use without cost or licensing issues. So many problems we'll face have already been solved in Python and we can use those solutions as we need.

Python has many packages that are tools for the scientific, mathematical and engineering field. We will generally refer to these fields under the general title "numerical."

\section{Using this document}

Occasionally, code, both as it would appear in a source file and as it would appear in an interpreter or terminal, appears in the text. Such code should be easy to distinguish from the text as it will be formated in a monospace font \verb|like so|. Filenames and directories will be \textbf{bolded}.

\section{Tentative Topic Overview}

The topics of this book are grouped together into four chapters. Each chapter has several units and each unit attempts to correspond to a once a week, two hour lecture with homework assignments. 

\begin{enumerate}

\subsection{Introduction to Python}

    \item Introduction and Overview of Python, its capabilities and basic installation, setup and invoking the CPython interpreter. Discussion of development environments and the flavors of Python (2 vs 3, Jython, Pypy, Iron Python, CPython )

    \item Control Flow, Functions and Types

    \item I/O, Exceptions and Classes. Operator overloading.

    \item Overview of the Python Standard Library. Python Package Index (PyPI).

\subsection{Scientific and Engineering Applications and Tools}
    \item Overview of scientific uses for Python. Overview of Sci/Eng tools in Python: SciPy, Numpy, guiqwt, matplotlib, Spyder, Sage, SimPy, PyMC, rpy2, OpenOpt, FiPy. Focus on matplotlib. Spring/Mass/Damper example problem with matplotlib. 

    \item SciPy and Numpy and more matplotlib (mplot3d, videos, histograms )

    \item Augmented interpreters: IPython, Spyder, Sage, VPython.

    \item Sympy (mpmath, units, CAS stuff).

\subsection{Software development in Python}
    \item Smaller tasks (scripting). Useful modules: optparse/argparse, re module, shutil, pickle, subprocess, csv, cmd

    \item GUIs. Overview and emphasis on a particular framework. Probably going to go with tk since its the builtin. Traits, Chaco.

    \item Debuggers and analysis: pylint, profile, cprofile, pdb and winpdb

    \item Profiling and Optimization in Python. Profiling: profile, cprofile. Optimized types: collections module, xrange, list comprehension over filter(). Call functions from current namespace rather than module namespace. Numpy vectors rather than loops. Preallocate lists? Optimized interpreters: Pypy.

    \item Larger tasks: setuptools, version control overview, xml module, pydoc, sphinx, test module, readline, fileinput.

\subsection{Advanced topics.}
    \item PyOpenCl

    \item Concurrency and parallelism: the threading and multiprocessing modules

    \item Interfacing with other languages: ctypes, Boost.Python, Jython, f2py, IronPython

    \item Pypy and RPython

    \item Python frameworks: twisted, django, Qt.

    \item Python and the Internet

    \item Advanced optimization: Cython, weave

\end{enumerate}


\chapter{Introduction To Python}
\section{Python Overview and Applications}
Python has a wide variety of uses. \emph{Fill these in later}
\subsection{Interpreters and Environments}

Python is generally an interpreted language\footnote{There are dialects that are compiled and several different types of compilers. Also, the interpreter byte compiles Python expressions as part of its normal operation.}, so Python programs are usually executed as they are read from the source file. The interpeter also allows Python to be used interactively, in a \gls{repl}. 

In addition to the standard Python interpreter distributed by the Python Software Foundation, there are several alternative interpreters that add additional functionality. IPython, in particular, is used extensively in the numerical community. Everything that works in the standard Python interpreter should work in IPython, so it can be used in this chapter instead of the standard Python interpreter. IPython is also part of the Spyder \gls{ide}, which we will use later.

Python files are ordinary text files that are interpreted by the Python interpreter.  As such, they can be created in any text editor. Additionally, there are \glspl{ide} that integrate the text editor with useful tools such as an interpreter, a debugger or class browser. Later, we will investigate a numerically oriented \gls{ide}, Spyder. 

\subsection{Reading}
Topics 1-3 in the Python Tutorial at \url{http://docs.python.org/tutorial/} \cite{website:Python272docs}
\subsection{Problems}
Install Python on your computer. Use version 2 as that's what we'll be using for this class: \url{http://python.org/download/}

\subsection{Control Flow, Functions and Types}
Class 1 covers control flow statements such as \verb|if|, \verb|else|, \verb|for| and \verb|while|; sequence data types including lists, strings and tuples; dictionaries; more looping techniques including list comprehensions and functional tools like \verb|map()| and \verb|filter()|; and Python modules. We do not cover sets, thought they are discussed in the Tutorial.

The reading and assignment together may take as much as four hours.

\subsection{Reading}
Topics 4-6 in the Python Tutorial at \url{http://docs.python.org/tutorial/} \cite{website:Python272docs}
\subsection{Problems}
\begin{enumerate}
	\item Write a for loop that prints the numbers 1 through 50 and prints "fizz" if the number is divisible by 2 and "buzz" if it's divisible by 3.

The first few numbers should look like so:
\begin{verbatim}
1
2 Fizz
3 Buzz
4 Fizz
5
6 Fizz Buzz
\end{verbatim}
etc.

	\item Explain what the \verb|continue| statement does.
	\item Write a function, \verb|mysum()|, that takes a list as an argument and returns the sum of the elements in the list.
	\item Write a function, \verb|map()|, that takes a function, \verb|f|, as it's first argument, a list as it's second argument and returns a list of the results of applying f to each element in the argument list. So \verb|mymap( f, [1,2,3] )|

would return a list such that

\verb|[f(1), f(2), f(3)]|.

	\item Write a function, \verb|list_args()|, that takes an arbitrary number of arguments and returns its arguments in a list.
	\item Modify the function from five to return just the number of arguments when passed the keyword argument
\verb|mode="counter"|

	\item Write a function, \verb|empty()|, that takes an object and a list with an arbitrary number of elements equal to that object as arguments and removes all of those elements from the list.

	\item Write a version of \verb|empty()| that doesn't alter the input list, but instead returns a version of the input list with all of the elements equal to to the object removed.

	\item Write a version of either \verb|empty()| that uses the \verb|filter()| function.

	\item Write a function \verb|pnorm()|, that takes a list of numbers and a non-zero positive integer, \verb|p|, and returns the p-norm of the list as if it were a vector. \verb|p| should have a default value of 2. Use the \verb|map()| and \verb|reduce()| functions to implement this.

\emph{Hint: The }\verb|pow()| \emph{function is in the math standard module.}

	\item Implement \verb|pnorm()| from 4 using list comprehensions instead of \verb|map()|.

	\item Implement \verb|pnorm()| from 4, except allow p to be a tuple of non-zero positive integers. For p=(1,2,3,x,y,z...), \verb|pnorm()| should return a tuple of the norms, such that (1-norm, 2-norm, 3-norm, x-norm, y-norm, z-norm...).

	\item Create a dictionary \verb|team_locations| with keys that are the American League West team names (Athletics, Mariners,Rangers and Angels of Anaheim) and the values are their respective locations (Oakland, Seattle, Texas and Los Angeles). Write a function \verb|exchange()|, that takes \verb|team_locations| and exchanges each key for its value and vice versa. So:

\begin{verbatim}
>>> output = exchange( team\_locations)
>>> print output[Seattle]
\end{verbatim}
gives

\verb|Mariners|
	

	\item Create a module using a directory named \textbf{first} using \textbf{\_\_init\_\_.py}. Create a directory named \textbf{second} in \textbf{first} and make it a module as well. Finally, create a file \textbf{lecture1.py} that contains the definitions of \verb|mymap()|, \verb|list\_args()|, \verb|mysum()|, \verb|empty()|, \verb|pnorm()| and \verb|exchange()| as well as the \verb|team_locations| dict and make it a submodule of \textbf{second} so:

\verb|import first.second.lecture1|

imports all of the functions except \verb|exchange()|. If and only if \textbf{lecture1.py} is run as an executable, it should print the 2-norm of \verb|[1,1,1]|.

\end{enumerate}

\section{Class 2: Input/Output, Exceptions and Classes}
All useful computer languages have input/output facilities. Python's represent a good mix of power and ease of use. Exceptions provide a powerful and customizable method of error checking that also produces useful debugging information that's accessible from the command line or from within the interpreter.

Python is an object oriented language -- objects bind together data and functions to make designing and managing programs easier as well as making it easy to reuse code. They also allow the user to define new data types, such as matrices, that can use Python's native operators like addition and multiplication. To implement user types, the user defines a \gls{class} which contains the object's data and functions that use and/or alter that data.

Python has two types of classes, old-style and new-style. While the tutorial uses old-style classes, users should use new style classes. The only difference we need to worry about is when declaring a class. When declaring a class that doesn't inherit from any other class, new-style classes inherit from object. So when the tutorial defines a class with \verb|class <Classname>:| we should use \verb|class <Classname>(object):|.

\subsection{Reading}
Topics 7-9 in the Python Tutorial at \url{http://docs.python.org/tutorial/} \cite{website:Python272docs}
\subsection{Problems}
\begin{enumerate}
	\item Explain the difference between \verb|str()| and \verb|repr()|
	\item Write a function, \verb|file_reverser()|, that takes the name of an input file and an output file, reads in the input file and writes out the input file to the output file line by in reverse. The function should close both files before returning.

See attached files \textbf{input.txt} and \textbf{line\_reversed.txt} for example input files and reversed files.
	\item Write a function \verb|pickler()| that uses pickle to store a list or dictionary to file and a function \verb|unpickler()| that returns the dictionary or list that was stored given the file name.
	\item Add exception handling to the \verb|unpickler()| function from 4 so that when given a filename that doesn't exist, it returns \verb|None| rather than allowing an exception through.
	\item Modify the \verb|pickler()| function from 4 to raise an exception if the object passed to be pickled is neither a list nor a dictionary. The exception should be a \verb|TypeError| and error message should be \verb|"Silly rabbit, dicts are for pickling"|

\emph{Hint: }\verb|isinstance()| \emph{and} \verb|type()|

	\item Define a class \verb|Door| that has one state, \verb|opened|, that can be true or false. Define a methods \verb|close()| and \verb|open()| for \verb|Door| that alter the \verb|opened| state. 
	\item Define a class \verb|MyArray| that takes a set of numeric arguments and stores them in order. Define the \verb|__repr__()| and \verb|__str__()| methods for the class such that:

\begin{verbatim}
>>>array = MyArray(arg1, arg2, arg3)
>>>repr(MyArray)
MyArray(arg1, arg2, arg3)}
\end{verbatim}

and \verb|print( MyArray)| gives
\verb|[ arg1, arg2, arg3 ]|

Also define the \verb|__add__()| and \verb|__len__()| methods so that \verb|+| implements piecewise addition and \verb|len()| returns the length of the array.

So:
\begin{alltt}
>>>bob = MyArray(1,2,3) + MyArray(3,2,1)
>>>print(bob)
[4,4,4]
\end{alltt}

and \verb|len(MyArray(1,2,3)| returns:
\verb|3|

	\item Define a class \verb|MyMatrix| that inherits from \verb|MyArray| but also takes row and column lengths. The \verb|__str__()| and \verb|__repr__()| methods should be overloaded so that \verb|str()| would print a 2x2 matrix like so

\begin{verbatim}
[1 0
 0 1]
\end{verbatim}

and \verb|repr()| would return

\verb|"MyMatrix( 2,2, 1, 0, 0, 1 )"|

\end{enumerate}

\section{ Batteries Included: Standard Library and the Python Package Index Overview}

One of Python's fundamental goals is to be a "batteries included" language -- to have as much functionality built into the core language as reasonably possible. As a result, the Python Standard Library is both large and very flexible. 

Python packages can also be downloaded from their project websites; often hosted at public sourcecode repositories like Sourceforge, Google Code and Github; or found by your operating system's \gls{package manager}. The Linux distributions Ubuntu and Debian use \href{http://wiki.debian.org/Aptitude?action=show&redirect=aptitude}{aptitude}, \href{http://wiki.debian.org/Apt}{apt-get} or \href{https://help.ubuntu.com/community/SynapticHowto}{Synaptic}, Fedora and Redhat use \href{http://yum.baseurl.org/}{yum} and OS X users can install \href{http://www.finkproject.org/}{Fink}, which uses apt-get as well, or \href{http://www.macports.org/index.php}{MacPorts}. Package managers make it easy to search for packages by name and description.

Additionally, there is an offical website for cataloging and sharing user created Python packages, the Python Package Index or "PyPI" (not to be confused with PyPy which is, among other things, an optimizing Python interpreter. We will come back to it later.) PyPI can also be accessed through the \verb|easy_install| and \verb|pip| packages.

\subsection{Reading}
Topics 10-13 of the Python tutorial. "Installing Distributions from the Python Package Index" from the \href{http://wiki.python.org/moin/CheeseShopTutorial}{PyPI tutorial}. 

\subsection{Problems}

\begin{enumerate}
	\item What standard library module provides file wildcards like '*'?
	\item What module enables reading and writing to variable length binary formats?
	\item Write a function, \verb|sqrtn()| that calculates the square root of its first argument to \verb|n| decimal places, where \verb|n| is its second argument, using the \verb|decimal| module.
	\item What does \verb|sys.argv| contain?
	\item Write a command program that takes prints the square root of its first argument to the precisions given by its second argument.
	\item Install \href{http://matplotlib.sourceforge.net/}{matplotlib}, a plotting package for Python. Instructions here: \url{http://matplotlib.sourceforge.net/users/installing.html}
\end{enumerate}

\chapter{Scientific, Mathematical and Engineering Tools}
\section{Overview}
Python is increasingly popular for numerical computing, the primary focus of scientific, mathematical and engineering computing. The core language has decent support for numerical computing via its \verb|math| module and its extensionable type system. Even better, however, is the strong numerical support in the Python community and the excellent packages that exist as a result. The best known of these are the SciPy suite maintained by Enthought and the Sage mathematical suite developed by mathematicians such as William Stein at the University of Washington.

SciPy\cite{website:scipy} includes the NumPy array and matrix module for fast calculations on arrays of floating point data, the matplotlib plotting package and a variety of subpackages for common numerical tasks such as integration, linear algebra, signal processing, interpolation and optimization. We will use matplotlib\cite{website:matplotlib} shortly to plot the behavior of a simple spring-mass-damper system.

Spyder is an IDE that provides a Matlab-like environment that includes an IPython based interactive console, an object inspector, a variable explorer, a text editor and several other useful tools. We will use it for much of this chapter. Spyder is part of the \verb|spyderlib| module that allows users to imbed Spyder-like tools in their own applications.

Other packages expand Python's numerical capabilities even further:

\begin{description}
	\item[FiPy] provides a finite volume partial differential equation solver in Python. \url{http://www.ctcms.nist.gov/fipy/}
	\item[guiqwt] provides a 2D plotting system specialized for interactive plotting and signal and image processing. \url{http://packages.python.org/guiqwt/}
	\item[OpenOpt] provides a general optimization framework with the ability to call external solvers as well. \url{http://openopt.org/}
	\item[PyMC] implements Bayesian estimation using the Markov Chain Monte Carlo method. \url{http://code.google.com/p/pymc/}
	\item[rpy2] provides a Python interface to the stastical analysis oriented R programming language. \url{http://rpy.sourceforge.net/index.html}
	\item[SimPy] is a Python extension that adds a process-based discrete-event simulation framework. \url{http://simpy.sourceforge.net/}
	\item[sympy] is a library for symbolic mathematics. It includes the mpmath library which includes arbitrary precision numbers and interval arithmetic. \url{http://sympy.org/}
	\item[uncertainties] allows Python to track uncertainties in random variables across algebraic operations. It works with NumPy arrays. \url{http://rpy.sourceforge.net/index.html}
\end{description}


\subsection*{Note on Floating Point Calculations}
Since much of Python's numerical tools use floating point numbers (Python's \verb|float| type is an IEEE 754 binary64, also known as a "double"), it is important to note that binary floating point numbers do not exactly represent many decimal fractions. Much as $1/3$ cannot be represented by a finite number of digits in base 10, numbers such as $1/10$ cannot be represented exactly in a Python \verb|float|. This requires, among other things, that equality comparisons should be avoided when dealing with floating points. Instead, the values should considered equal if they are "close enough." In many cases, a simple \verb|if abs( value1 - value2 ) < tolerance:| is acceptible, but it is not a universal solution. For more indepth discussions, see \href{http://www.cygnus-software.com/papers/comparingfloats/comparingfloats.htm}{"Comparing Floating Point Numbers"}  or the in depth \href{http://www.cse.msu.edu/~cse320/Documents/FloatingPoint.pdf}{\emph{What Every Computer Scientist Should Know About Floating-Point}}.

\subsection{Reading}
The matplotlib introduction and installation instructions. Python Tutorial topic 14.

\subsection{The Spring Mass Damper, a Numerical and Plotting Example}

We will create a simple spring mass damper system with integrator in pure Python and plot its behavior in matplotlib.

\subsubsection{Spring Mass Damper System}

Our test system will be a simple damped harmonic oscillator representing a spring-mass-damper system. Recall that forced, damped harmonic oscillators follow the equation\cite{website:wikipedia}

\begin{displaymath}
F(t)-kx-c\frac{dx}{dt}=m\frac{d^2x}{dt^2}
\end{displaymath}

where \(x\) is displacement, \(\frac{dx}{dt}\) is velocity \(\frac{d^2x}{dt^2}\) is acceleration, \(m\) is the mass, \(k\) is the spring constant, \(c\) is the damping constant, \(t\) is time and \(F\) is the forcing function. Traditionally, we solve such systems by treating them has two first order differential equations:

\begin{displaymath}
\frac{dx_0}{dt} = x_1
\end{displaymath}

\begin{displaymath}
\frac{dx_1}{dt} =  \frac{F(t)-kx-c x_1}{m}
\end{displaymath}


We can create a class that stores all of that information and has a method \verb|derivatives()| that returns a function that gives the system derivatives \(\frac{dx_0}{dt}\) and \(\frac{dx_1}{dt}\) as a function of time and state:

\begin{verbatim}
class HarmonicOscillator(object):
  """A class that characterizes spring mass damper systems"""
  def __init__(self, mass, spring, damping=0.0, forcing = None):
    """Initialize the system. Damping defaults to zero and the forcing function 
        defaults to None.""
    self.mass = mass
    self.spring = spring
    self.damping = damping
    if forcing == None:
      self.forcing = lambda x: 0.0
    else:
      self.forcing = forcing

  def acceleration(self, time, velocity, displacement):
    """Returns the acceleration of the system as a function of time, 
    velocity and displacement"""
    mass = self.mass
    accel = self.forcing(time)/mass - self.spring/mass*displacement - \
                 self.damping/mass*velocity
    return accel

  def derivatives(self):
    """Returns an anonymous function that gives the state derivaties of the system"""
    return lambda time, state: state[2], self.acceleration( time, state[1], state[2] )
      
\end{verbatim}

\subsubsection{Integration}

\subsubsection{Plotting the Results}

You should have matplotlib already installed.

\subsection{Problems}



\chapter{Recommended Resources}
\section{Python 2.7}
\begin{description}
	\item[Tutorials]
Python 2.7.2 Tutorial: \url{http://docs.python.org/tutorial/index.html} \cite{website:Python272docs}
	\item[Library References]
Python 2.7.2 Standard Library:  \url{http://docs.python.org/library/} \cite{website:Python272docs}
	\item[Books]
\emph{Learning Python, 3rd Edtion}: \url{http://oreilly.com/catalog/9780596513986}  \cite{LearningPython3rd}
\end{description}

\chapter{Extended Resources}
\section{Educational Resources}
\begin{description}
	\item[Tutorials]

Python 3.2 Tutorial: \url{http://docs.python.org/py3k/tutorial/index.html}  \cite{website:Python3docs}
	\item[Classes]

Google Python class: \url{http://code.google.com/edu/languages/google-python-class/index.html} \cite{website:GooglePythonClass}

MIT Open Course Ware Introduction to Computer Science and Programming:

\href{http://ocw.mit.edu/courses/electrical-engineering-and-computer-science/6-00-introduction-to-computer-science-and-programming-fall-2008/}{http://ocw.mit.edu/courses/electrical-engineering-and-computer-science/6-00-introduction-to-computer-science-and-programming-fall-2008/} \cite{website:OCW}

\emph{A Python Book: Beginning Python, Advanced Python, and Python Exercises}: \url{http://www.rexx.com/~dkuhlman/python_book_01.html}

	\item[Books]
\emph{Learning Python, 4th ed} (Python 3): \url{http://oreilly.com/catalog/9780596158071}  \cite{LearningPython4th}

\emph{Programming Python}: \url{http://oreilly.com/catalog/9780596009250}  \cite{ProgrammingPython} 

\section{Communities}
	\item[Wikis]
PythonInfo: \url{http://wiki.python.org/moin/}
	\item[Aggregators]
Python Reddit: \url{http://python.reddit.com}
Stack Overflow: \url{http://stackoverflow.com/questions/tagged/python}

\end{description}

\printglossaries

\bibliographystyle{plain}
\bibliography{citations}

\end{document}