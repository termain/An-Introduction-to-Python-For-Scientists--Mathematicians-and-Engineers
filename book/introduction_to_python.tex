
\chapter{Introduction To Python}
\section{Python Overview and Applications}
Python has a wide variety of uses. \emph{Fill these in later}
\subsection{Interpreters and Environments}

Python is generally an interpreted language\footnote{There are dialects that are compiled and several different types of compilers. Also, the interpreter byte compiles Python expressions as part of its normal operation.}, so Python programs are usually executed as they are read from the source file. The interpeter also allows Python to be used interactively, in a \gls{repl}. 

In addition to the standard Python interpreter distributed by the Python Software Foundation, there are several alternative interpreters that add additional functionality. IPython\cite{ipython}, in particular, is used extensively in the numerical community. Everything that works in the standard Python interpreter should work in IPython, so it can be used in this chapter instead of the standard Python interpreter. IPython adds system shell access, numbered prompts with command history, file system navigation, object introspection, tab completion, session logging that can be reused as Python code and allow for session restoration and more to the standard Python interpreter. IPython is also part of the Spyder \gls{ide}, which we will use later.

Python files are ordinary text files that are interpreted by the Python interpreter.  As such, they can be created in any text editor. Additionally, there are \glspl{ide} that integrate the text editor with useful tools such as an interpreter, a debugger or class browser. Later, we will investigate a numerically oriented \gls{ide}, \gls{Spyder}.\cite{website:spyder} 

\subsection{Reading}
Topics 1-3 in the Python Tutorial at \url{http://docs.python.org/tutorial/} \cite{website:Python272docs}
\subsection{Problems}

\begin{enumerate}
  \item Install Python on your computer. Use version 2 as that's what we'll be using for this class: \url{http://python.org/download/}
  \item Install IPython on your computer. \url{http://ipython.scipy.org/doc/stable/html/install/install.html}
\end{enumerate}

\section{Control Flow and Functions}
He we cover control flow statements such as \verb|if|, \verb|else|, \verb|for| and \verb|while| as well as defining and using functions in Python. sequence data types including lists, strings and tuples; dictionaries; more looping techniques including list comprehensions and functional tools like \verb|map()| and \verb|filter()|; and Python modules. We do not cover sets, thought they are discussed in the Tutorial.

The reading and assignment together may take as much as four hours.

\subsection{Reading}
Topic 4 in the Python Tutorial at \url{http://docs.python.org/tutorial/} \cite{website:Python272docs}

\subsection{Function example: nth roots}
Here we define a nth root function.  We allow the user to set the tolerance of the method, but set it to a default value of 1.0e-6. We also set the default value of n to 2.

First we declare the function using \verb|def| and name its arguments inside parentheses. We have a mandatory argument (\verb|argument|) and optional arguments \verb|nn|, the \emph{n} in nth root, and \verb|tolerance|, the maximum difference that between two successive values that acts as our convergence criterion. When two successive values are less than \verb|tolerance| apart, we return the latter of the two values as our approximation of the nth root. The end of the function declaration is marked with a colon.

Below the function declaration, we indent one level and begin the function body. The first line of the function body is the \gls{docstring}. Docstrings are triple quoted strings that contain the documentation for the function, class, method or module that is displayed when Python's \verb|help()| function is called on that object.

The actual function computation begins after the docstring and still indented one level. Since the Babylonian method is an iterative method of finding roots, we first set \verb|old_value| to our initial guess (in this case, \verb|argument|), and we iterate once to get \verb|new_value|. Note that Python's single comment character is the pound sign, \verb|#|. Multiple line comments (and multiline strings) use the triple quote, \verb|"""|. Only a triple quoted string beginning directly after function declaration is a docstring, however.

Next comes a |while| loop, which will repeat until the condition inside its parenthesis is met. If the condition is met before the while loop is run (as is the case when \verb|argument| is 1, for instance), then the body of the loop will be skipped. In this case, each successive iteration of the loop will make the absolute value of the differences in the iteration results, \verb|abs(old_value - new_value)| smaller, so eventually the loop will exit, unless \verb|tolerance| is too close to or below the tolerance of the datatype of \verb|argument|, in which case the successive values may never get close enough to exit the loop. 

Once the loop has exited, \verb|nth_root()| returns the final value in \verb|new_value| as an approximation of nth root.

\begin{lstlisting}[caption=Nth root function,
  label=nthrootfunction,
  float=h!]
def nth_root( argument, nn = 2, tolerance = 1e-6 ):
    """Finding the nth Root using the Babylonian method"""
    old_value = argument #We use the argument as our initial guess.
    new_value = (  (nn -1)*old_value + argument/old_value**(nn-1)  )/nn

    while(  abs(old_value - new_value ) < tolerance ):
        old_value = new_value
        new_value = (  (nn -1)*old_value + argument/old_value**(nn-1)  )/nn

    return( new_value )

\end{lstlisting}

%add later: An example that uses a guessing function for the first rough estimate of the nth root as an example of passing functions as arguments to other functions.

\subsection{Problems}
\begin{enumerate}
	\item Write a for loop that prints the numbers 1 through 50 and prints "fizz" if the number is divisible by 2 and "buzz" if it's divisible by 3.

The first few numbers should look like so:
\begin{verbatim}
1
2 Fizz
3 Buzz
4 Fizz
5
6 Fizz Buzz
\end{verbatim}
etc.

	\item Explain what the \verb|continue| statement does.
	\item Write a function, \verb|mysum()|, that takes a list as an argument and returns the sum of the elements in the list.
\end{enumerate}	

\section{Sequence types and dictionaries}
\subsection{Reading}
Topic 5 in the Python Tutorial at \url{http://docs.python.org/tutorial/} \cite{website:Python272docs}
\subsection{Problems}
\begin{enumerate}
	\item Write a function, \verb|map()|, that takes a function, \verb|f|, as it's first argument, a list as it's second argument and returns a list of the results of applying f to each element in the argument list. So \verb|mymap( f, [1,2,3] )|

would return a list such that

\verb|[f(1), f(2), f(3)]|.

	\item Write a function, \verb|list_args()|, that takes an arbitrary number of arguments and returns its arguments in a list.
	\item Modify the function from five to return just the number of arguments when passed the keyword argument
\verb|mode="counter"|

	\item Write a function, \verb|empty()|, that takes an object and a list with an arbitrary number of elements equal to that object as arguments and removes all of those elements from the list.

	\item Write a version of \verb|empty()| that doesn't alter the input list, but instead returns a version of the input list with all of the elements equal to to the object removed.

	\item Write a version of either \verb|empty()| that uses the \verb|filter()| function.

	\item Write a function \verb|pnorm()|, that takes a list of numbers and a non-zero positive integer, \verb|p|, and returns the p-norm of the list as if it were a vector. \verb|p| should have a default value of 2. Use the \verb|map()| and \verb|reduce()| functions to implement this.

\emph{Hint: The }\verb|pow()| \emph{function is in the math standard module.}

	\item Implement \verb|pnorm()| from 4 using list comprehensions instead of \verb|map()|.

	\item Implement \verb|pnorm()| from 4, except allow p to be a tuple of non-zero positive integers. For p=(1,2,3,x,y,z...), \verb|pnorm()| should return a tuple of the norms, such that (1-norm, 2-norm, 3-norm, x-norm, y-norm, z-norm...).

	\item Create a dictionary \verb|team_locations| with keys that are the American League West team names (Athletics, Mariners,Rangers and Angels of Anaheim) and the values are their respective locations (Oakland, Seattle, Texas and Los Angeles). Write a function \verb|exchange()|, that takes \verb|team_locations| and exchanges each key for its value and vice versa. So:

\begin{verbatim}
>>> output = exchange( team\_locations)
>>> print output[Seattle]
\end{verbatim}
gives

\verb|Mariners|
\end{enumerate}	

\section{Modules}
\subsection{Reading}
Topic 6 in the Python Tutorial at \url{http://docs.python.org/tutorial/} \cite{website:Python272docs}

\subsection{Problems}
\begin{enumerate}	

	\item Create a module using a directory named \textbf{first} using \textbf{\_\_init\_\_.py}. Create a directory named \textbf{second} in \textbf{first} and make it a module as well. Finally, create a file \textbf{lecture1.py} that contains the definitions of \verb|mymap()|, \verb|list\_args()|, \verb|mysum()|, \verb|empty()|, \verb|pnorm()| and \verb|exchange()| as well as the \verb|team_locations| dict and make it a submodule of \textbf{second} so:

\verb|import first.second.lecture1|

imports all of the functions except \verb|exchange()|. If and only if \textbf{lecture1.py} is run as an executable, it should print the 2-norm of \verb|[1,1,1]|.

\end{enumerate}

\section{Input/Output}
%, Exceptions and Classes}
All useful computer languages have input/output facilities. Python's represent a good mix of power and ease of use.

\subsection{Reading}
Topic 7 in the Python Tutorial at \url{http://docs.python.org/tutorial/} \cite{website:Python272docs}

\subsection{Problems}
\begin{enumerate}
	\item Explain the difference between \verb|str()| and \verb|repr()|
	\item Write a function, \verb|file_reverser()|, that takes the name of an input file and an output file, reads in the input file and writes out the input file to the output file line by in reverse. The function should close both files before returning.

See attached files \textbf{input.txt} and \textbf{line\_reversed.txt} for example input files and reversed files.
	\item Write a function \verb|pickler()| that uses pickle to store a list or dictionary to file and a function \verb|unpickler()| that returns the dictionary or list that was stored given the file name.
\end{enumerate}

\section{Exceptions}
 Exceptions provide a powerful and customizable method of error checking that also produces useful debugging information that's accessible from the command line or from within the interpreter.

\subsection{Reading}
Topic 8 in the Python Tutorial at \url{http://docs.python.org/tutorial/} \cite{website:Python272docs}

\subsection{Problems}

\begin{enumerate}

	\item Add exception handling to the \verb|unpickler()| function from 4 so that when given a filename that doesn't exist, it returns \verb|None| rather than allowing an exception through.
	\item Modify the \verb|pickler()| function from 4 to raise an exception if the object passed to be pickled is neither a list nor a dictionary. The exception should be a \verb|TypeError| and error message should be \verb|"Silly rabbit, dicts are for pickling"|

\emph{Hint: }\verb|isinstance()| \emph{and} \verb|type()|
\end{enumerate}

\section{Classes}
Python is an object oriented language -- objects bind together data and functions to make designing and managing programs easier as well as making it easy to reuse code. They also allow the user to define new data types, such as matrices, that can use Python's native operators like addition and multiplication. To implement user types, the user defines a \gls{class} which contains the object's data and functions that use and/or alter that data.

Python has two types of classes, old-style and new-style. While the tutorial uses old-style classes, users should use new style classes. The only difference we need to worry about is when declaring a class. When declaring a class that doesn't inherit from any other class, new-style classes inherit from object. So when the tutorial defines a class with \verb|class <Classname>:| we should use \verb|class <Classname>(object):|.

\subsection{Reading}
Topic 9 in the Python Tutorial at \url{http://docs.python.org/tutorial/} \cite{website:Python272docs}
\subsection{Problems}
\begin{enumerate}
	\item Define a class \verb|Door| that has one state, \verb|opened|, that can be true or false. Define a methods \verb|close()| and \verb|open()| for \verb|Door| that alter the \verb|opened| state. 
\end{enumerate}
\subsection{MyVector Class Example}

\lstinputlisting[caption=MyVector class,
label=myvector, firstline=1, lastline=35,
float=ht!]{$PYTHONFORSMELOCATION/code/introduction_to_python/examples.py}

Vectors are extremely useful data types and pervade physical system modeling. We can use classes to define our own vector type so we can set its behavior exactly as we see fit. Here, we uses Python classes to define a vector class that is can be constructed from any numeric type that supports addition, multiplication and division. Operator overloading allows us to dictate that \verb|+|, \verb|-| and \verb|*| work exactly as they should for vectors.

Like functions, classes are declared with a reserved word, \verb|class|, and the class body follows on indented lines. Unlike functions, the objects listed in parentheses are not arguments, but instead are the classes the new class inherits from. We are using Python 2 new-style classes, so all classes that inherit from nothing else inherit from \verb|object|. After the class declaration, the class's docstring should be next, followed by the class's default members and methods.

The first method defined for a class is generally the \verb|__init__()| method, which is called when an object of the class is instantiated. This method is the class's default constructor. The \verb|MyVector| class will need a list at intialization which it can convert into a \verb|MyVector|.

Notice that the first argument of \verb|__init__()| is \verb|self|. All of the class's methods will take \verb|self| as their first argument in their definitions. \verb|self| represents the object itself and any member of \verb|self| that a method changes is changed in the object, even after the method returns.

Next, we define the \verb|__len__()| method. It is called by Python's builtin \verb|len()| function so that calling \verb|len( myObject )| will return the same value as \verb|myObject.__len__()| if \verb|myObject| has \verb|__len__()| defined. Not surprisingly, it returns the length of the vector.

Now, since we know how long the vector is, we can easily define addition. Vector addition is defined as "element i of the sum, c, is equal to the sum of element i of the addend a and element i of the addend b for all\footnote{\begin{math}\forall\end{math} is the mathematical symbol for "for all"} integer i from zero to n-1 where n is the length of the vectors a and b." Or, more concisely:

\begin{displaymath}
c_i = a_i+b_i, \forall i=0...n-1
\end{displaymath}

i, of course, is the index of the vector element. Python has two methods that let us access the elements of a sequence object like a list or MyVector directly if we know the index, instead of having to access them through the object's member variables. These are \verb|__getitem__()| and \verb|__setitem__()|. \verb|__getitem__(self, index)| allows one to access the vector's (or any class's) element at \verb|index|. \verb|__setitem__(self,index, value)| sets the element at \verb|index| to \verb|value|.  Once these methods have defined for a class, an object of that class's elements can be accessed using square brackets, just like lists.

Now we can easily define addition for the \verb|MyVector| type. We will define two similar methods, \verb|__add__()| and \verb|__radd__()|, which overload the \verb|+| operator. First, we make sure that our two addends, \verb|self| and \verb|other| are the same length. Note that \verb|self| is an addend because \verb|__add__()| method of the \verb|MyVector| class. By default, Python calls the \verb|__add__()| method of the left argument to the \verb|+| sign. If it doesn't find one, it calls the \verb|__radd__()| method of the right argument to \verb|+|. If that is not available, the interpreter will raise an exception.

Since lists and tuples have \verb|__getitem__()| and \verb|__setitem()__| defined as well, they  can be passed as the \verb|other| argument to \verb|__add__()|. Python will happily attempt to add them element by element to the elements of the calling \verb|MyVector| so long as the individual elements can be added to the elements of \verb|MyVector|.

If the lengths of \verb|self| and \verb|other| are different, we we raise an exception. Python has good one already defined, \verb|ArithmeticError|. Next we create our sum vector from a list of zeros of \verb|self|'s length. This way, we use \verb|vector_sum[index]=self[index]+other[index]| for each element since we've already defined the \verb|__getitem__()| and \verb|__setitem__()| methods for \verb|MyVector|.

Now we can use the \verb|range()| function to generate a list of indices that we loop over, summing each element of the addends and putting the result in \verb|sum| as we go. Once we've looped over all of the elements, we return \verb|sum|, a new \verb|MyVector| object.

For completeness, we define \verb|__radd__()| by calling \verb|other|'s \verb|__add__()| method with  \verb|self| as the second argument. Remember, should \verb|__radd__()| get called, for instance if the left operand of \verb|+| is a list, the \verb|self| argument will refer to the right operand of \verb|+|.

\verb|MyVector| is starting to resemble a useful tool. We can construct \verb|MyVector|s from lists and tuples, add two \verb|MyVectors| with lists and tuples as well as other \verb|MyVectors| and access their individual elements by index. However, when we try and print a \verb|MyVector|, we get something like this: 

\verb|>>>print MyVector([1,2,3])|

\verb|<__main__.MyVector object at 0x1287ed0>|

Python doesn't yet know how to display a \verb|MyVector|. Fortunately, that's what the \verb|__str__()| method is for. It returns a string representation of the object. The \verb|print| statement automatically calls \verb|__str__| if asked to print the object. The \verb|__repr__()| can also be defined. It returns a string that is used by the interpreter to represent an object, such as what the intepreter displays when one simply enters the object's name without assigning it to anything.

\lstinputlisting[caption=MyVector \_\_str\_\_ method,
label=myvector, firstline=37, lastline=49,
float=ht!]{$PYTHONFORSMELOCATION/code/introduction_to_python/examples.py}

So, now the \verb|MyVector| is printed just like a list would be:

\verb|>>>print MyVector([1,2,3])|

\verb|[1,2,3]|

Without \verb|__repr__()|:

\verb|>>>MyVector([1,2,3])|

\verb|<__main__.MyVector object at 0xd15f10>|

With \verb|__repr__()| defined (it just returns \verb|str(verb)|):

\verb|>>>MyVector([1,2,3])|

\verb|[1,2,3]|

\subsection{Problems}

\begin{enumerate}

	\item Define a class \verb|MyMatrix| that inherits from \verb|MyVector| but also takes row and column lengths. The \verb|__str__()| and \verb|__repr__()| methods should be overloaded so that \verb|str()| would print a 2x2 matrix like so

\begin{verbatim}
[1 0
 0 1]
\end{verbatim}

and \verb|repr()| would return

\verb|"MyMatrix( 2,2, 1, 0, 0, 1 )"|

\end{enumerate}

\section{ Batteries Included: Standard Library and the Python Package Index Overview}

One of Python's fundamental goals is to be a "batteries included" language -- to have as much functionality built into the core language as reasonably possible. As a result, the Python Standard Library is both large and very flexible. 

Python packages can also be downloaded from their project websites; often hosted at public sourcecode repositories like Sourceforge, Google Code and Github; or found by your operating system's \gls{package manager}. The Linux distributions Ubuntu and Debian use \href{http://wiki.debian.org/Aptitude?action=show&redirect=aptitude}{aptitude}, \href{http://wiki.debian.org/Apt}{apt-get} or \href{https://help.ubuntu.com/community/SynapticHowto}{Synaptic}, Fedora and Redhat use \href{http://yum.baseurl.org/}{yum} and OS X users can install \href{http://www.finkproject.org/}{Fink}, which uses apt-get as well, or \href{http://www.macports.org/index.php}{MacPorts}. Package managers make it easy to search for packages by name and description.

Additionally, there is an offical website for cataloging and sharing user created Python packages, the Python Package Index or "PyPI"\footnote{Not to be confused with PyPy which is, among other things, an optimizing Python interpreter. We will come back to it later.} PyPI can also be accessed through the \verb|easy_install|\footnote{\url{http://peak.telecommunity.com/DevCenter/EasyInstall}} and \verb|pip|\footnote{\url{http://www.pip-installer.org/en/latest/index.html}} packages.

\subsection{Reading}
Topics 10-13 of the Python tutorial.  

\subsection{Problems}

\begin{enumerate}
	\item What standard library module provides file wildcards like '*'?
	\item What module enables reading and writing to variable length binary formats?
	\item Write a function, \verb|sqrtn()| that calculates the square root of its first argument to \verb|n| decimal places, where \verb|n| is its second argument, using the \verb|decimal| module.
	\item What does \verb|sys.argv| contain?
	\item Write a command program that prints the square root of its first argument to the precisions given by its second argument.
	\item Install \href{http://matplotlib.sourceforge.net/}{matplotlib}, a plotting package for Python. Instructions here: \url{http://matplotlib.sourceforge.net/users/installing.html}

\end{enumerate}